\documentclass[11pt,landscape,a4paper,fleqn]{article}
\usepackage[utf8]{inputenc}
\usepackage[british]{babel}
\usepackage{tikz}
\usetikzlibrary{shapes,positioning,arrows,fit,calc,graphs,graphs.standard}
\usepackage{multicol}
\usepackage{wrapfig}
\usepackage[top=4mm,bottom=5mm,left=4mm,right=4mm]{geometry}
\usepackage[framemethod=tikz]{mdframed}
\usepackage{microtype}
\usepackage{mathtools}
\usepackage{ccicons}
\usepackage{hyperref}
\usepackage{lmodern}
\PassOptionsToPackage{dvipsnames}{xcolor}
\usepackage[dvipsnames]{xcolor}
\usepackage{soul}
\usepackage{amssymb}
\usepackage{amsmath}
\usepackage[neveradjust]{paralist}
\usepackage[shortlabels]{enumitem}
\usepackage{bbm}
\usepackage{listings}
\usepackage{libertine}
\usepackage[libertine]{newtxmath}
% \usepackage{ETbb}
% \usepackage[sc]{mathpazo}
\usepackage{algpseudocode}
\usepackage{inconsolata}
\usepackage{physics}
\usepackage{bm}
\usepackage{svg}
\usepackage{wrapfig}
\usepackage{titlesec}
\usepackage{booktabs}

\providetoggle{showextratext}
\settoggle{showextratext}{false}

\setlength{\columnsep}{2mm}

\setlist{topsep=0pt, leftmargin=*, noitemsep, topsep=0pt,parsep=0pt,partopsep=0pt}

\newcommand{\extratext}[1]{\iftoggle{showextratext}{#1}{}}

\newcommand{\todo}[1]{\textcolor{red}{\textbf{TODO:} #1}}

\newcommand*{\tran}{^{\mathsf{T}}} % (DIN) EN ISO 80000-2:2013
\newcommand{\kl}[2]{D_{\mathrm{KL}}(#1\lVert#2)}
\newcommand{\js}[2]{D_{\mathrm{JS}}(#1\lVert#2)}

\newcommand{\E}{\mathbb{E}}
\DeclareMathOperator{\softmax}{\mathrm{Softmax}}
\DeclareMathOperator{\diag}{\mathrm{diag}}
\newcommand{\R}{\mathbb{R}}
\renewcommand{\exp}{\mathrm{exp}}
\DeclareMathOperator*{\argmax}{arg\,max}
\DeclareMathOperator*{\argmin}{arg\,min}

\newcommand{\sep}{\,\textcolor{red}{\bm|}\,}

\setdefaultleftmargin{0.5cm}{}{}{}{}{}

\let\bar\overline

\definecolor{myblue}{cmyk}{1,.72,0,.38}
\definecolor{myorange}{cmyk}{0,0.5,1,0}
\definecolor{myorange2}{cmyk}{0,0.8,0.8,0}
\definecolor{darkgreen}{cmyk}{0.97,0,1,0.57}

\pgfdeclarelayer{background}
\pgfsetlayers{background,main}

%\everymath\expandafter{\the\everymath \color{myblue}}
%\everydisplay\expandafter{\the\everydisplay \color{myblue}}

\renewcommand{\baselinestretch}{.8}
\pagestyle{empty}

\setlength{\parindent}{0pt}

\def\myvector#1{\mathbf{#1}}
\def\va{{\myvector{a}}}
\def\vb{{\myvector{b}}}
\def\vc{{\myvector{c}}}
\def\vd{{\myvector{d}}}
\def\ve{{\myvector{e}}}
\def\vf{{\myvector{f}}}
\def\vg{{\myvector{g}}}
\def\vh{{\myvector{h}}}
\def\vi{{\myvector{i}}}
\def\vj{{\myvector{j}}}
\def\vk{{\myvector{k}}}
\def\vl{{\myvector{l}}}
\def\vm{{\myvector{m}}}
\def\vn{{\myvector{n}}}
\def\vo{{\myvector{o}}}
\def\vp{{\myvector{p}}}
\def\vq{{\myvector{q}}}
\def\vr{{\myvector{r}}}
\def\vs{{\myvector{s}}}
\def\vt{{\myvector{t}}}
\def\vu{{\myvector{u}}}
\def\vv{{\myvector{v}}}
\def\vw{{\myvector{w}}}
\def\vx{{\myvector{x}}}
\def\vy{{\myvector{y}}}
\def\vz{{\myvector{z}}}

\def\mymatrix#1{\mathbf{#1}}
\def\mA{{\mymatrix{A}}}
\def\mB{{\mymatrix{B}}}
\def\mC{{\mymatrix{C}}}
\def\mD{{\mymatrix{D}}}
\def\mE{{\mymatrix{E}}}
\def\mF{{\mymatrix{F}}}
\def\mG{{\mymatrix{G}}}
\def\mH{{\mymatrix{H}}}
\def\mI{{\mymatrix{I}}}
\def\mJ{{\mymatrix{J}}}
\def\mK{{\mymatrix{K}}}
\def\mL{{\mymatrix{L}}}
\def\mM{{\mymatrix{M}}}
\def\mN{{\mymatrix{N}}}
\def\mO{{\mymatrix{O}}}
\def\mP{{\mymatrix{P}}}
\def\mQ{{\mymatrix{Q}}}
\def\mR{{\mymatrix{R}}}
\def\mS{{\mymatrix{S}}}
\def\mT{{\mymatrix{T}}}
\def\mU{{\mymatrix{U}}}
\def\mV{{\mymatrix{V}}}
\def\mW{{\mymatrix{W}}}
\def\mX{{\mymatrix{X}}}
\def\mY{{\mymatrix{Y}}}
\def\mZ{{\mymatrix{Z}}}

\titleformat{\section}{\color{myorange}\sffamily\normalsize}{\thesection}{0.5em}{}
\titleformat{\subsection}[runin]{\color{myorange2}\sffamily\small}{\thesubsection}{0.5em}{}
\titleformat{\subsubsection}[runin]{\color{violet}\sffamily\small}{\thesubsubsection}{0.5em}{}

\titlespacing*{\section}
{0pt}{0.3em}{0em}
\titlespacing*{\subsection}
{0pt}{0.3em}{1em}
\titlespacing*{\subsubsection}
{0pt}{0.1em}{1em}

\begin{document}
\setlength{\columnseprule}{0.1pt}
	
% \section*{Disclaimer}

\begin{multicols*}{4}
% \setlength{\columnseprule}{0.0pt}

% This document is a summary of the \textit{Machine Perception} course at ETH Z\"urich.
% This summary was created during the spring semester of 2024.
% Due to updates to the syllabus content, some material may no longer be relevant for future versions of the course.
% I do not guarantee correctness or completeness, nor is this document endorsed by the lecturers.
% The order of the chapters is not necessarily the order in which they were presented in the course.
% For the full \LaTeX \ source code, visit \texttt{\href{https://github.com/Jovvik/eth-cheatsheets}{github.com/Jovvik/eth-cheatsheets}}.

% All figures are created by the author, and, assuming the rules have not been changed,
% are not allowed to be used as a part of a cheat sheet during the exam.

% \newpage
\setlength{\columnseprule}{0.1pt}
\section{ML}

Perceptron converges in finite time iff data is linearly separable.
\textbf{MAP} $\theta^* \in \argmax p(\theta \mid X, y)$.
\textbf{MLE} $\theta \in \argmax p(y \mid X, \theta)$ consistent, efficient.
\textbf{Binary cross-entropy} $L(\theta) = - y_i \log(\hat{y}_i) - (1 - y_i) \log(1 - \hat{y}_i)$.
Cross-entropy $H(p_d, p_m) = H(p_d) + \kl{p_d}{p_m}$.
For any continuous $f$ $\exists \mathrm{NN}\ g(x), |g(x) - f(x)| < \varepsilon$.
1 hidden layer is enough, activation function needs to be nonlinear.

MLP backward inputs: $\delta^{(l)} = \delta^{(l + 1)} \cdot \pdv{\vz^{(l + 1)}}{\vz^{(l)}}$,
backward weights: $\left[\pdv{\vz^{(l)}}{\vw_{ij}^{(l)}}\right]_k = f'(\va)_k \cdot z_j^{(l)} \cdot [k = i],
\pdv{L}{\vz^{(l)}}{\vz^{(l)}} = \delta^{(l)} \pdv{\vz^{(l)}}{\vw_{i,j}^{(l)}}$,
backward bias: $\pdv{L}{\vb_i^{(l)}} =$ same, but no $\vz$.

\subsection{Activation functions} \phantom{a}

% \renewcommand{\arraystretch}{0}
\begin{tabular}[]{@{}c@{}ccc@{}}
    \toprule
    name & $f(x)$ & $f'(x)$ & $f(X)$ \\
    \midrule
    sigmoid & $\frac{1}{1 + e^{-x}}$ & $\sigma(x)(1 - \sigma(x))$ & $(0, 1)$ \\
    tanh & $\frac{e^x - e^{-x}}{e^x + e^{-x}}$ & $1 - \tanh(x)^2$ & $( - 1, 1)$ \\
    ReLU & $\max(0, x)$ & $[x \geq 0]$ & $[0, \infty)$ \\
\end{tabular}
Finite range: stable training, mapping to prob. space.
Sigmoid, tanh saturate (value with large mod have small gradient) $ \Rightarrow $ vanishing gradient,
Tanh is linear around 0 (easy learn),
ReLU can blow up activation; piecewise linear $ \Rightarrow $ faster convergence.

\subsection{GD algos}

\textbf{SGD}: use 1 sample.
For sum structured loss is unbiased.
High variance, efficient, jumps a lot $ \Rightarrow $ may get out of local min.,
may overshoot.
\textbf{Mini-batch}: use $m < n$ samples.
More stable, parallelized.
\textbf{Polyak's momentum}: velocity $\vv \coloneqq \alpha \vv - \epsilon \nabla_\theta L(\theta), \theta \coloneqq \theta + \vv$.
Move faster when high curv., consistent or noisy grad.
\textbf{Nesterov's momentum}: $\vv \coloneqq \alpha \vv - \epsilon \nabla_\theta L(\theta + \alpha \vv)$.
Gets grad. at future point.
\textbf{AdaGrad}: $\vr \coloneqq \vr + \nabla \odot \nabla$, $\Delta \theta = - \epsilon / (\delta + \sqrt{\vr})
\odot \nabla$.
Grads decrease fast for variables with high historical gradients,
slow for low. But can decrease LR too early/fast.
\textbf{RMSProp}: $\vr \coloneqq \rho \vr + (1 - \rho) \nabla \odot \nabla$,
use weighted moving average $ \Rightarrow $ drop history from distant past,
works better for nonconvex.
\textbf{Adam}: collect 1st and 2nd moments:
$\vm \coloneqq \beta_1 \vm + (1 - \beta_1) \nabla, \vv \coloneqq \beta_2 \vv + (1 - \beta_2) \nabla \odot \nabla$,
unbias:
$\hat{\vm} = \vm / (1 - \beta_1^t), \hat{\vv} = \vv / (1 - \beta_2^t)$,
$\Delta \theta =- \frac{\eta}{\sqrt{\hat{\vv}} + \epsilon}\hat{\vm}$.


\section{CNN} $T$ is linear if $T(\alpha \vu + \beta \vv) = \alpha T(\vu) + \beta T(\vv)$,
invariant to $f$ if $T(f(\vu)) = T(\vu)$,
equivariant to $f$ if $T(f(\vu)) = f(T(\vu))$.
Any linear shift-equivariant $T$ can be written as a convolution.
Convolution:
$I'(i, j) = \sum_{m =- k}^k \sum_{n =- k}^k K(\textcolor{red}{-}m, \textcolor{red}{-}n)I(m + i, n + j)$.
Correlation:
$I'(i, j) = \sum_{m =- k}^k \sum_{n =- k}^k K(m, n)I(m + i, n + j)$.
Conv. forward: $z^{(l)} = w^{(l)} * z^{(l - 1)} + b^{(l)} = \sum_m \sum_n w_{m, n}^{(l)} z_{i - m, j - n}^{(l - 1)} + b^{(l)}$.
Backward inputs: $\delta^{(l - 1)} = \pdv{C}{z_{i,j}^{(l - 1)}} = \delta^{(l)} * \mathrm{ROT}_{180}(w^{(l)})$,
backward kernel: $\pdv{C}{w_{m, n}^{(l)}} = \delta^{(l)} * \mathrm{ROT}_{180}(z^{(l - 1)})$.
Width or height after conv or pool: $(\mathrm{in} + 2 \cdot \mathrm{pad} - \mathrm{dil} \cdot (\mathrm{kern} - 1) - 1) / \mathrm{stride} + 1$,
rounded down.
Channels = number of kernels.

1D conv as matmul:
\(\begin{bmatrix}
    k_1 & 0 & \dots & 0 \\
    k_2 & k_1 &  & \vdots \\
    k_3 & k_2 & k_1 & 0 \\
    0 & k_3 & k_2 & 0 \\
    \vdots & \vdots & \ddots & \vdots \\
\end{bmatrix}\)

Backprop example (\textcolor{red}{rotate $\mK$}):
\[
    \stackrel{\begin{bmatrix}
        0 & 0 & 1 & 1 & 0 \\
        0 & 1 & 0 & 1 & 1 \\
        1 & 1 & 1 & 1 & 0 \\
        1 & 0 & 1 & 1 & 0 \\
        1 & 1 & 0 & 1 & 1
    \end{bmatrix}}{\mX} \to
    \stackrel{\begin{bmatrix}
        1 & 0 & 1 \\
        0 & 1 & 0 \\
        0 & 1 & 0 
    \end{bmatrix}}{\mK} \to
    \stackrel{\begin{bmatrix}
        3 & 3 & 3 \\
        4 & 2 & 3 \\
        2 & 3 & 3 
    \end{bmatrix}}{\mY = \mX * \mK} \to
\]
\[
    \to\stackrel{\begin{bmatrix}
        4
    \end{bmatrix}}{\mY' = \mathrm{Pool}(\mY)}
    \sep
    \stackrel{\begin{bmatrix}
        1
    \end{bmatrix}}{\partial E / \partial \mY'} \to
    \stackrel{\begin{bmatrix}
        0 & 0 & 0 \\
        1 & 0 & 0 \\
        0 & 0 & 0 \\
    \end{bmatrix}}{\partial E / \partial \mY} \to
\]
\[
    \to\stackrel{\begin{bmatrix}
        1 & 0 & 1 \\
        1 & 1 & 1 \\
        0 & 1 & 0 \\
    \end{bmatrix}}{\partial E / \partial \mK} \to
    \stackrel{\begin{bmatrix}
        0 & 0 & 0 & 0 & 0 \\
        0 & 1 & 0 & 0 & 0 \\
        0 & 1 & 0 & 0 & 0 \\
        1 & 0 & 1 & 0 & 0 \\
        0 & 0 & 0 & 0 & 0
    \end{bmatrix}}{\partial E / \partial \mV}
\]

Max-pooling: $z^{(l)} = \max z_i^{(l - 1)}$. $i^* \coloneqq \argmax_i z_i^{(l - 1)}$,
$\pdv{z^{(l)}}{z_i^{(l - 1)}} = [i = i^*]$, $\delta^{(l - 1)} = \delta^{(l)}_{i^*}$.
Unpooling: nearest-neighbor (duplicate), bed of nails (only top left, rest 0), max-unpooling (remember where max came from when pooling).
Learnable upsampling: transposed conv, output is copies of filter weighted by input, summed on overlaps.

\textbf{Sizes}
input: $(C_{in}, H_{in}, W_{in})$, output: $(C_{out}, H_{out}, W_{out})$.
Kernel: $k_1 \times k_2$, padding: $p_1 \times p_2$, stride: $s_1 \times s_2$, dilation: $d_1 \times d_2$.
$H_{out} = (H_{in} + 2p_1 - d_1(k_1 - 1) - 1) s^{-1}_1 + 1$,
$W_{out} = (W_{in} + 2p_2 - d_2(k_2 - 1) - 1) s^{-1}_2 + 1$.

\section{RNN}

\textbf{Vanilla \textsc{rnn}}: $\hat{y}_t = \mW_{hy} \vh_t, \vh_t = f(\vh_{t-1}, \vx_t, \mW)$
, usually $\vh_t = \tanh(\mW_{hh} \vh_{t-1} + \mW_{xh}\vx_t)$.

\textbf{BPTT}: $\pdv{L}{\mW} = \sum_t \pdv{L_t}{\mW}$, treat unrolled model as multi-layer.
$\pdv{L_t}{W}$ has a term of $\pdv{\vh_t}{\vh_k} = \prod_{i = k + 1}^t \pdv{\vh_i}{\vh_{i-1}}
= \prod_{i = k + 1}^t \mW_{hh}\tran \diag f'(\vh_{i-1})$.

\textbf{Exploding/vanishing gradients}: $\vh_t = \mW^t \vh_1$.
If $\mW$ is diagonaliz., $\mW = \mQ \diag \bm\lambda \mQ\tran = \mQ \bm\Lambda \mQ\tran, \mQ \mQ\tran = \mI
\Rightarrow \vh_t = (\mQ\bm\Lambda \mQ\tran)^t \vh_1 = (\mQ (\diag \bm\lambda)^t \mQ\tran) \vh_1$
$\Rightarrow \vh_t$ becomes the dominant eigenvector of $\mW$.
$\pdv{\vh_t}{\vh_k}$ has this issue.
Long-term contributions vanish, too sensitive to recent distrations.
\textbf{Truncated BPTT}: take the sum only over the last $\kappa$ steps.
\textbf{Gradient clipping} $\frac{\mathrm{threshold}}{\norm{\nabla}} \nabla$ fights exploding gradients.


\subsection{LSTM}
We want constant error flow, not multiplied by $W^t$.
\begin{itemize}
    \item Input gate: which values to write,
    \item forget gate: which values to reset,
    \item output gate: which values to read,
    \item gate: candidate values to write to state.
\end{itemize}

\begin{minipage}{0.3\linewidth}
    \begin{align*}
        \begin{pmatrix}
            \vi \\
            \vf \\
            \vo \\
            \vg
        \end{pmatrix} & = \begin{pmatrix}
            \sigma \\
            \sigma \\
            \sigma \\
            \tanh
        \end{pmatrix} W \begin{pmatrix}
            \vx_t \\
            \vh_{t-1}
        \end{pmatrix} \\
        \vc_t & = \vf \odot \vc_{t-1} + \vi \odot \vg \\
        \vh_t & = \vo \odot \tanh(\vc_t)
    \end{align*}
\end{minipage}%
%\begin{minipage}{0.4\linewidth}
    %\includesvg[width=\textwidth]{figures/LSTM.svg}
%\end{minipage}


\section{Generative modelling}

Learn $p_{\mathrm{model}} \approx p_{\mathrm{data}}$, sample from $p_{\mathrm{model}}$.

\begin{itemize}[leftmargin=0.5em]
    \item Explicit density:
    \begin{itemize}[leftmargin=0.3em]
        \item Approximate:
        \begin{itemize}[leftmargin=0.0em]
            \item Variational: VAE, Diffusion
            \item \textcolor{gray}{Markov Chain: Boltzmann machine}
        \end{itemize}
        \item Tractable:
        \begin{itemize}[leftmargin=0.0em]
            \item Autoregressive: FVSBN/NADE/MADE, Pixel\textsc{(c/r)nn}, WaveNet/\textsc{tcn}, Autor. Transf., 
            \item Normalizing Flows
        \end{itemize}
    \end{itemize}
    \item Implicit density:
    \begin{itemize}[leftmargin=0.3em]
        \item Direct: Generative Adversarial Networks
        \item \textcolor{gray}{MC: Generative Stochastic Networks}
    \end{itemize}
\end{itemize}

Autoencoder: $X \textcolor{blue}{\to} Z \textcolor{red}{\to} X$, $\textcolor{red}{g} \circ \textcolor{blue}{f} \approx \mathrm{id}$,
$f$ and $g$ are NNs. Optimal linear autoencoder is PCA.

Undercomplete: $|Z| < |X|$, else overcomplete.
Overcomp. is for denoising, inpainting.

Latent space should be continuious and interpolable.
Autoencoder spaces are neither,
so they are only good for reconstruction.

\section{Variational AutoEncoder (VAE)}

Sample $z$ from prior $p_\theta(z)$, to decode use conditional $p_\theta(x \mid z)$ defined by a NN.

$\kl{P}{Q} \coloneqq \int_x p(x) \log \frac{p(x)}{q(x)} \dd x$: KL divergence,
measure similarity of prob. distr.

$\kl{P}{Q} \neq \kl{Q}{P}, \kl{P}{Q} \geq 0$

$z$ can also be categorical.
Likelihood $p_\theta(x) = \int_z p_\theta(x \mid z) p_\theta(z) \dd z$ is hard to maximize,
let encoder NN be $q_\phi(z \mid x)$,
$\log p_\theta(x^{i}) = \textcolor{orange}{\E_z\left[\log p_\theta(x^{i} \mid z)\right]}
- \textcolor{purple}{\kl{q_\phi(z \mid x^{i})}{p_\theta(z)}} + \textcolor{red}{\kl{q_\phi(z \mid x^{i})}{p_\theta(z \mid x^{i})}}$.
\textcolor{red}{Red} is intractable, use $\geq 0$ to ignore it;
\textcolor{orange}{Orange} is reconstruction loss, clusters similar samples;
\textcolor{purple}{Purple} makes posterior close to prior, adds cont. and interp.
$\mathrm{\textcolor{orange}{Orange}} - \mathrm{\textcolor{purple}{Purple}}$ is \textbf{ELBO}, maximize it.

$x \xrightarrow{\mathrm{\makebox[0pt]{\scriptsize \textcolor{blue}{enc}}}} \mu_{z \mid x}, \Sigma_{z \mid x} \xrightarrow{\mathrm{sample}} z \xrightarrow{\mathrm{\makebox[0pt]{\scriptsize \textcolor{red}{dec}}}} \mu_{x \mid z}, \Sigma_{x \mid z}  \xrightarrow{\mathrm{sample}} \hat{x}$

Backprop through sample by reparametr.: $z = \mu + \sigma \epsilon$.
For inference, use $\mu$ directly.

Disentanglement: features should correspond to distinct factors of variation.
Can be done with semi-supervised learning by making $z$ conditionally independent of given features $y$.

\subsection{$\beta$-VAE}

$\max_{\theta, \phi} \E_x\left[\E_{z \sim q_\phi} \log p_\theta(x \mid z)\right]$
to disentangle s.t.
$\kl{q_\phi(z \mid x)}{p_\theta(z)} < \delta$, with KKT: $\max \textcolor{orange}{\mathrm{Orange}} - \beta\textcolor{purple}{\mathrm{Purple}}$.

\section{Autoregressive generative models}

Autoregression: use data from the same input variable at previous time steps

Discriminative: $P(Y \mid X)$, generative: $P(X, Y)$, maybe with $Y$ missing.
Sequence models are generative: from $x_i \dots x_{i + k}$ predict $x_{i + k + 1}$.

Tabular approach: $p(\vx) = \prod_i p(x_i \mid \vx_{<i})$, needs $2^{i - 1}$ params.
Independence assumption is too strong.
Let $p_{\theta_i}(x_i \mid \vx_{ < i}) = \operatorname{Bern}(f_i(\vx_{ < i}))$,
where $f_i$ is a NN.
\textbf{Fully Visible Sigmoid Belief Networks}: $f_i = \sigma(\alpha^{(i)}_0 + \bm{\alpha}^{(i)} \vx_{ < i}\tran)$,
complexity $n^2$, but model is linear.

\textbf{Neural Autoregressive Density Estimator}: add hidden layer.
$\vh_i = \sigma(\vb + \mW_{\centerdot, < i} \vx_{ < i})$,
$\hat{x}_i = \sigma(c_i + \mV_{i,\centerdot} \vh_i)$.
Order of $\vx$ can be arbitrary but fixed.
Train by max log-likelihood in $\mathcal{O}(TD)$, can use 2nd order optimizers,
can use \textbf{teacher forcing}: feed GT as previous output.

Extensions: Convolutional; Real-valued: conditionals by mixture of gaussians;
Order-less and deep: one DNN predicts $p(x_k \mid x_{i_1} \dots x_{i_j})$.

\textbf{Masked Autoencoder Distribution Estimator}:
mask out weights s.t. no information flows from $x_d \dots $ to $\hat{x}_d$.
Large hidden layers needed.
Trains as fast as autoencoders, but sampling needs $D$ forward passes.

\textbf{PixelRNN}: generate pixels from corner, dependency on previous pixels is by RNN (LSTM).
\textbf{PixelCNN}: also from corner, but condition by CNN over context region (perceptive field) $ \Rightarrow $ parallelize.
For conditionals use masked convolutions.
Channels: model R from context, G from R + cont., B from G + R + cont.
Training is parallel, but inference is sequential $ \Rightarrow $ slow.
Use conv. stacks to mask correctly.

NLL is a natural metric for autoreg. models,
hard to evaluate others.

\textbf{WaveNet}: audio is high-dimensional.
Use dilated convolutions to increase perceptive field with multiple layers.

AR does not work for high res images/video, convert the images into a series of tokens with an AE:
Vector-quantized VAE.
The codebook is a set of vectors.
$x \xrightarrow{\mathrm{\makebox[0pt]{\scriptsize \textcolor{blue}{enc}}}} z \xrightarrow{\mathrm{codebook}} z_q \xrightarrow{\mathrm{\makebox[0pt]{\scriptsize \textcolor{red}{dec}}}} \hat{x}$.

We can run an AR model in the latent space.

\subsection{Attention}

$\vx_t$ is a convex combination of the past steps, with access to all past steps.
For $X \in \R^{T \times D}$: 
$K = XW_K, V = XW_V, Q = XW_Q$.
Check pairwise similarity between query and keys via dot product:
let attention weights be $\bm{\alpha} = \softmax(QK\tran / \sqrt{D})$, $\bm{\alpha} \in \R^{1 \times T}$.
Adding mask $M$ to avoid looking into the future:
\[X = \softmax\left(\frac{(XW_Q)(XW_K)\tran}{\sqrt{D}} + M\right)(XW_V)\]
Multi-head attn. splits $W$ into $h$ heads, then concatenates them.
Positional encoding injects information about the position of the token.
Attn. is $\mathcal{O}(T^2 D)$.

\section{Generative Adversarial Networks (GANs)}

Log-likelihood is not a good metric. We can have high likelihood with poor quality by mixing in noise and not losing much likelihood; or low likelihood with good quality by remembering input data and having sharp peaks there.

\textbf{Generator} $G : \R^Q \to \R^D$ maps noise $z$ to data,
\textbf{discriminator} $D : \R^D \to [0, 1]$ tries to decide if data is real or fake,
receiving both gen. outputs and training data.
Train $D$ for $k$ steps for each step of $G$.

Training GANs is a min-max process, which are hard to optimize.
$V(G, D) = \E_{\vx \sim p_{\mathrm{d}}} \log(D(\vx)) + \E_{\hat{\vx} \sim p_{\mathrm{m}}} \log(1 - D(\hat{\vx}))$

For $G$ the opt. $D^* = p_{\mathrm{d}}(\vx) / (p_{\mathrm{d}}(\vx) + p_{\mathrm{m}}(\vx))$.

Jensen-Shannon divergence (symmetric):
$\js{p}{q} = \frac{1}{2} \kl{p}{\frac{p + q}{2}} + \frac{1}{2} \kl{p}{\frac{p + q}{2}}$.
Global minimum of $\js{p_{\mathrm{d}}}{p_{\mathrm{m}}}$ is the glob. min. of $V(G, D)$,
$V(G, D^*) = - \log(4)$ and at optimum of $V(D^*, G)$ we have $p_d = p_m$.

If $G$ and $D$ have enough capacity, at each update step $D$ reaches $D^*$
and $p_{\mathrm{m}}$ improves $V(p_{\mathrm{m}}, D^*) \propto \sup_D \int_{\vx} p_{\mathrm{m}}(\vx) \log( - D(\vx)) \dd \vx$,
then $p_{\mathrm{m}} \to p_{\mathrm{d}}$ by convexity of $V(p_{\mathrm{m}}, D^*)$ wrt. $p_{\mathrm{m}}$.
These assumptions are too strong.

If $D$ is too strong, $G$ has near zero gradients and doesn't learn ($\log'(1 - D(G(z))) \approx 0$).
Use gradient \underline{ascent} on $\log(D(G(z)))$ instead.

\textbf{Mode collapse}: $G$ only produces one sample or one class of samples.
Solution: \textbf{unrolling} --- use $k$ previous $D$ for each $G$ update.

GANs are hard to compare, as likelihood is intractable.
FID is a metric that calculates the distance between feature vectors
calculated for real and generated images.

%DCGAN: pool $\to$ strided convolution, batchnorm, no FC, ReLU for $G$, LeakyReLU for $D$.

%Wasserstein GAN: different loss, gradients don't vanish.
Adding gradient penalty for $D$ stabilizes training.
\textbf{Profressive Growing} GAN: generate low-res image, then high-res during training.
\textbf{AdaIN}: Similar to attention. $c' = \gamma(s) \odot \frac{c - \mu (c)}{\sigma(c)} + \beta(s)$, where $c$ is the content, $s$ is the style.
\textbf{StyleGAN}: learn intermediate latent space $\mathcal{W}$ with FCs,
batchnorm with scale and mean from $\mathcal{W}$, add noise at each layer using AdaIn.

GAN \textbf{inversion}: find $z$ s.t. $G(z) \approx x$ $ \Rightarrow $ manipulate images in latent space, inpainting.
If $G$ predicts image and segmentation mask,
we can use inversion to predict mask for any image, even outside the training distribution.

\subsection{3D GANs}

3D GAN: voxels instead of pixels.
PlatonicGAN: 2D input, 3D output differentiably rendered back to 2D for $D$.

HoloGAN: 3D GAN + 2D superresolution GAN

GRAF: radiance fields more effic. than voxels

GIRAFFE: GRAF + 2D conv. upscale

EG3D: use 3 2D images from StyleGAN for features, project each 3D point to tri-planes.

\subsection{Image Translation}

E.g. sketch $X \to$ image $Y$.
Pix2Pix:
$G : X \to Y$,
$D : X, Y \to [0, 1]$.
GAN loss $+ L_1$ loss between sketch and image.
Needs pairs for training.

CycleGAN: unpaired.
Two GANs $F: X \to Y, G : Y \to X$,
cycle--consistency loss $F \circ G \approx \mathrm{id}; G \circ F \approx \mathrm{id}$
plus GAN losses for $F$ and $G$.

BicycleGAN: add noise input.

Vid2vid: video translation.

\section{Normalizing Flows}

VAs dont have a tractable likelihood, AR models have no latent space.
Want both.
Change of variable for $x = f(z)$:
$p_x(x) = p_z(f^{-1}(x)) \abs{\det \pdv{f^{-1}(x)}{x}} = p_z(f^{-1}(x)) \abs{\det \pdv{f(z)}{z}}^{-1}$.
Map $Z \to X$ with a deterministic invertible $f_\theta$.
This can be a NN, but computing the determinant is $\mathcal{O}(n^3)$.
If the Jacobian is triangular, the determinant is $\mathcal{O}(n)$.
To do this, add a coupling layer:
\begin{minipage}{0.5\linewidth}
    \[\begin{pmatrix}
        y^A \\
        y^B
    \end{pmatrix} = \begin{pmatrix}
        h(x^A, \beta(x^B)) \\
        x^B
    \end{pmatrix}\]
\end{minipage}%
\hspace{0.3cm}%
\begin{minipage}{0.45\linewidth}
    , where $\beta$ is any model, and $h$ is elementwise.
\end{minipage}
\[\begin{pmatrix}
    x^A \\
    x^B
\end{pmatrix} = \begin{pmatrix}
    h^{-1}(y^A, \beta(y^B)) \\
    y^B
\end{pmatrix}, J = \begin{pmatrix}
    h' & h'\beta' \\
    0 & 1
\end{pmatrix}\]
Stack these for expressivity, $f = f_k \circ \dots f_1$.
$p_x(x) = p_z(f^{-1}(x)) \prod_k \abs{\det \pdv{f_k^{-1}(x)}{x}}$.

Sample $z \sim p_z$ and get $x = f(z)$.

%\includesvg[width=\linewidth]{figures/flow.svg}

% \begin{wrapfigure}{R}{0.2\linewidth}
    %\includesvg[width=\linewidth]{figures/flow-block.svg}
% \end{wrapfigure}
\textbullet\ Squeeze: reshape, increase chan.

\textbullet\ ActNorm: batchnorm with init. s.t. output $\sim \mathcal{N}(0, \mI)$ for first minibatch.
    $\vy_{i,j} = \vs \odot \vx_{i,j} + \vb$,
    $\vx_{i, j} = (\vy_{i,j} - \vb) / \vs$,
    $\log\det = H \cdot W \cdot \sum_i \log \abs{\vs_i}$: linear.

\textbullet\ $1 \times 1$ conv: permutation along channel dim.
    Init $\mW$ as rand. ortogonal $\in \R^{C \times C}$ with $\det\mW = 1$.
    $\log\det = H \cdot W \cdot \log\abs{\det\mW}$: $\mathcal{O}(C^3)$.
    Faster: $\mW \coloneqq \mP \mL(\mU + \diag(s))$,
    where $\mP$ is a random \underline{fixed} permut. matrix,
    $\mL$ is lower triang. with 1s on diag.,
    $\mU$ is upper triang. with 0s on diag.,
    $\vs$ is a vector.
    Then $\log\det = \sum_i \log \abs{\vs_i}$: $\mathcal{O}(C)$
Conditional coupling: add parameter $\vw$ to $\beta$.

\textbf{SRFlow}: use flows to generate many high-res images from a low-res one.
Adds affine injector between conv. and coupling layers.
$\vh^{n+1} = \exp(\beta^n_{\theta, s}(\vu)) \cdot \vh^n + \beta_{\theta, b}(\vu)$,
$\vh^n = \exp( - \beta^n_{\theta, s}(\vu)) \cdot (\vh^{n+1} - \beta^n_{\theta, b}(\vu))$,
$\log\det = \sum_{i,j,k} \beta^n_{\theta, s}(\vu_{i, j, k})$, where $\vu = g_\Theta(x)$ is the low res image.

\textbf{StyleFlow}: Take StyleGAN and replace the network $\vz \to \vw$ (aux. latent space)
with a normalizing flow conditioned on attributes generated from the image.

\textbf{C-Flow}: condition on other normalizing flows: multimodal flows.
Encode original image $\vx_B^1$: $\vz_B^1 = f^{-1}_\phi(\vx_B^1 \mid \vx_A^1)$;
encode extra info (image, segm. map, etc.) $\vx_A^2$: $\vz_A^2 = g^{-1}_\theta(\vx_A^2)$;
generate new image $\vx_B^2$: $\vx_B^2 = f_\phi(\vz_B^1 \mid \vz_A^2)$.

Flows are expensive for training and low res.
The latent distr. of a flow needn't be $\mathcal{N}$.


\section{Diffusion models}

High quality generations, better diversity, more stable/scalable.

Diffusion (forward) step $q$: adds noise to $\vx_t$ (not learned).
Denoising (reverse) step $p_\theta$: removes noise from $\vx_t$ (learned).

$q(\vx_t \mid \vx_{t-1}) = \mathcal{N}(\sqrt{1 - \beta} \vx_{t-1}, \beta_t \mI)$

$p_\theta(\vx_{t-1} \mid \vx_t) = \mathcal{N}(\mu_\theta(\vx_t, t), \sigma_t^2 \mI)$

$\beta_t$ is the variance schedule (monotone $\uparrow$).
Let $\alpha_t \coloneqq 1 - \beta_t, \overline{\alpha}_t \coloneqq \prod \alpha_i$,
then $q(\vx_t \mid \vx_0) = \mathcal{N}(\sqrt{\overline{\alpha}_t} \vx_0, (1 - \overline{\alpha}_t)\mI)
\Rightarrow \vx_t = \sqrt{\overline{\alpha}_t} \vx_0 + \sqrt{1 - \overline{\alpha}_t}\epsilon$.

Denoising is not tractable naively:
$q(\vx_{t-1} \mid \vx_t) = q(\vx_t \mid \vx_{t-1}) q(\vx_{t-1}) / q(\vx_t)$,
$q(\vx_t) = \int q(\vx_t \mid \vx_0) q(\vx_0) \dd \vx_0$.

Conditioning on $\vx_0$ we get a Gaussian.
Learn model $p_\theta(\vx_{t-1} \mid \vx_t) \approx q(\vx_{t-1} \mid \vx_t, \vx_0)$
by predicting the mean.

$\log p(\vx_0) \geq \E_{q(\vx_{1:T} \mid \vx_0)} \log(\frac{p(\vx_{0:T})}{q(\vx_{1:T} \mid \vx_0)}) =
\textcolor{orange}{\E_{q(\vx_1 \mid \vx_0)}\log p_\theta(\vx_0 \!\!\mid\!\! \vx_1)} -
\textcolor{purple}{\kl{q(\vx_T \!\!\mid\!\! \vx_0)}{p(\vx_T)}} -
\textcolor{blue}{\sum_{t = 2}^T \E_{q(\vx_t \mid \vx_0)} \kl{q(\vx_{t-1} \!\!\mid\!\! \vx_t, \vx_0)}{p_\theta(\vx_{t-1} \!\!\mid\!\! \vx_t)}}$,
where \textcolor{orange}{orange} and \textcolor{purple}{purple} are the same as in VAEs,
and \textcolor{blue}{blue} are the extra loss functions.
In a sense VAEs are 1-step diffusion models.

$t$-th denoising is just $\argmin_\theta \frac{1}{2 \sigma_q^2(t)} \norm{\mu_\theta - \mu_q}_2^2$,
so we want $\mu_\theta(\vx_t, t) \approx \mu_q(\vx_t, \vx_0)$.
$\mu_q(\vx_t, \vx_0)$ can be written as
$\frac{1}{\sqrt{\alpha_t}} \vx_t - \frac{1 - \alpha_t}{\sqrt{1 - \overline{\alpha}_t} \sqrt{\alpha_t}} \epsilon_0$,
and $\mu_\theta(\vx_t, t) = \frac{1}{\sqrt{\alpha_t}} \vx_t - \frac{1 - \alpha_t}{\sqrt{1 - \overline{\alpha}_t} \sqrt{\alpha_t}} \hat{\epsilon}_\theta(\vx_t, t)$,
so the NN learns to predict the added noise.

Training: img $\vx_0, t \sim \mathrm{Unif}(1... T), \epsilon \sim \mathcal{N}(0, \mI)$,
GD on $\nabla_\theta\norm{\epsilon - \epsilon_\theta(\sqrt{\overline{\alpha}_t}\vx_0 + \sqrt{1 - \overline{\alpha}_t}\epsilon, t)}^2$.

Sampling: $\vx_T \sim \mathcal{N}(0, \mI)$, for $t = T$ downto $1$:
$\vz \sim \mathcal{N}(0, I)$ if $t > 1$ else $\vz = 0$;

$\vx_{t-1} = \frac{1}{\sqrt{\alpha_t}}(\vx_t - \frac{1 - \alpha_t}{\sqrt{1 - \overline{\alpha}_t}} \epsilon_\theta(\vx_t, t)) + \sigma_t \vz$.

$\sigma_t^2 = \beta_t$ in practice.
$t$ can be continuious.

$q(x_{t-1} | x_t, x_0) = \frac{x_t | x_{t-1}, x_0}{q(x_t | x_0)} \propto q(x_t | x_{t-1}, x_0)q(x_{t-1} | x_0)$


\subsection{Conditional generation}

Add input $y$ to the model.

\textbf{ControlNet}: don't retrain model, add layers that add something to block outputs. 
Use zero convolution to start with zero conditioning.

(Classifier-free) \textbf{guidance}: mix predictions of a conditional and unconditional model,
because conditional models are not diverse.
$\eta_{\theta_1}(x, c; t) = (1 + \rho)\eta_{\theta_1}(x, c; t) - \rho \eta_{\theta_2}(x; t)$.

\subsection{Latent diffusion models}

High-res images are expensive to model.
Predict in latent space, decode with a decoder.

\section{Foundation Models}
\textbf{Foundation models} are large pre-trained models that can be adapted to many tasks.
\textbf{First Generation}: Generalized Encoder + Task Decoder. Most Vision Models: ViT, MAE, SAM, CLIP, DINOv2. E.g. ELMO, BERT, ERNIE
\textbf{Second Generation}: Generalized Encoder + Task Finetuning. Diffusion Models: DreamBooth, Zero-1-to-3, SiTH. E.g. GPT-3
\textbf{Third Generation}: LLMs, can be prompted e.g. ChatGPT, LLaMa, Gemini, DeepSeek.
\textbf{ViT}: Vision Transformer, image as sequence of patches, transformer encoder.
\textbf{MAE}: Masked Autoencoder, encoder hallucinates missing patches, decoder reconstructs them.
\textbf{Sapiens}: MAE with 4 heads: 2d keypoints, segmentation, depth and normals.
\textbf{SAM}: Promptable segmentation. Supervised training. MAE + decoder conditioned on mask, points, box or text.
\textbf{CLIP}: Contrastive Language-Image Pretraining. Image and text encoded separately, contrastive learning put them close.
\textbf{DINOv2}: Self-supervised learning. Image encoder predicts image features from other images.
\textbf{DINO}: Student teacher distillation. Teacher knows more, gets small correction from student.
\textbf{4M}: Any-to-any multimodal model. Transformer encoder and decoder.
\textbf{DreamBooth}: Fine-tune diffusion model on a few images of a subject, then use it to generate images of that subject.
\textbf{Zero-1-to-3}: 3d reconstruction from single image. Condition on rotation and translation using CLIP embeddings.
\textbf{SiTH}: Single Image to High-Res. Condition diffusion on 3D model.

\section{Parametric body models}

\subsection{Pictorial structure}

Unary terms and pairwise terms between them with springs.

\subsection{Deep features}

Direct regression: predict joint coordinates with refinement.

Heatmaps: predict probability for each pixel, maybe Gaussian.
Can do stages sequentially.

\subsection{3D}

Naive 2D $\to$ 3D lift works.
But can't define constraints $ \Rightarrow $ 2m arms sometimes.

\textbf{Skinned Multi-Person Linear model} (SMPL) is the standard non-commerical model.
3D mesh, base mesh is $\sim$7k vertices, designed by an artist.
To produce the model, point clouds (scans) need to be aligned with the mesh.
\textbf{Shape deformation subspace}: for a set of human meshes T-posing,
vectorize their vertices $T$ and subtract the mean mesh.
With PCA represent any person as weighted sum of 10-300 basis people, $T = S\beta + \mu$.

For pose, use \textbf{Linear Blend Skinning}.
$\vt_i' = \sum_k w_{ki} \mG_k(\bm\theta, \mJ)\vt_i$,
where $\vt$ is the T-pose positions of vertices,
$\vt'$ is transformed, $w$ are weights,
$\mG_k$ is rigid bone transf.,
$\bm\theta$ is pose, $\mJ$ are joint positions.
Linear assumption produces artifacts.
\textbf{SMPL}:
$\vt_i' = \sum_k w_{ki} \mG_k(\bm\theta, \mJ(\bm\beta))(\vt_i + \vs_i(\bm\beta) + \vp_i(\bm\theta))$.
Adds shape correctives $\vs(\bm\beta) = \mS\bm\beta$,
pose cor. $\vp(\bm\theta) = \mP\bm\theta$,
$\mJ$ learned from shape $\bm\beta$ (least squares).

Predicting human pose is just predicting $\bm\beta, \bm\theta$ and camera parameters.

$\mathcal{W}$ are \textbf{skinning weights}, how vertices are influenced by joints.

$\mathcal{B}_S(\boldsymbol{\beta}, \mathcal{S}) = \sum_{n=1}^{|\boldsymbol{\beta}|}{\beta_n\mathbf{S}_n}$ is shape blend shape (identity of person).

$\mathcal{B}_P(\theta, P) = \sum_{n=1}^{9k}(R_n(\boldsymbol{\theta}) - R_n(\boldsymbol{\theta}^*))\mathbf{P}_n$ is pose blend shape aka pose correctives, $\mathbf{P}_n\in\mathbb{R}^{3N}$ blend shapes, learned from data, and $R(\theta): \mathbb{R}^{|\theta|}\rightarrow \mathbb{R}^{9K}$.

\textbf{SMPL Family}: MANO / SMPL+H, FLAME, SMPL-X, STAR.
\textbf{Human Mesh Recovery (HMR)}: predict $\bm\beta, \bm\theta$ from image. Projection + discriminator.
\textbf{SMPLify}: gradient = projection error + pose plausible in real life.
\textbf{LGD}: learn gradient descent, estimate gradient with NN, with actual gradients, current state and target 2D pose as input.
\textbf{EM-POSE}: use LGD on EM measurements.
\subsubsection{Optimization-based fitting} Predict 2D joint locations,
fit SMPL to them by argmin with prior regularization.
Argmin is hard to find, learn $F$: $\Delta \theta = F(\pdv{L_{reproj}}{\theta}, \theta^t, x)$.
Issues: self-occlusion, no depth info, non-rigid deformation (clothes).

\subsubsection{Template-based capture}
Scan for first frame, then track with SMPL.

\subsubsection{Animatable Neural Implicit Surfaces}
Model base shape and $w$ with 2 NISs. 

\section{Egocentric Vision}
FPV: + Dynamic uncurated content; + long-form video stream; + driven by goals, interactions and attention; - occlusions; - motion blur
TPV: + static; + controlled field of view; - curated moment in time; - disembodied. $\sep$ 1945: Bush's Memex, data storage. 2009: SixthSense, finger gestures. Gaze Tracker. 32 kitchens: object, action annotations, 55h.
Ego4D: 855 subjects, 3k h. 
hand and hand-object interactions: bounding boxes for hands and objects. DexYCB: videos grabbing things. HaMeR: ViT $\rightarrow$ Transformer head $\rightarrow$ MANO. 
$\sep$ action recognition and anticipation: past actions + task-relevant objects = action context. TransFusion: get context and predict location and next action and time to contact. PALM: prompt caption + action in LLM. 
$\sep$ gaze understanding and prediction: infer intention from gaze. PCI matches gaze changes. 
$\sep$  3D Scene Understanding: mask out active objects. LMK. EgoGaussian.
$\sep$ robotic applications: MAPLE.


\section{Neural Implicit Representations}

Voxels/volum. primitives are inefficient ($n^3$ compl.).
Meshes have limited granularity and have self-intersections.
\textbf{Implicit representation}: $S = \{x \mid f(x) = 0\}$.
Can be invertibly transformed without accuracy loss.
Usually represented as signed distance function values on a grid, but this is again $n^3$.
By UAT, approx. $f$ with NN.
\textbf{Occupancy networks}: predict probability that point is inside the shape.
\textbf{DeepSDF}: predict SDF.
Both conditioned on input (2D image, class, etc.).
NFs can model other properties (color, force, etc.).
$\sep$ Pro/cons: + easily combined + norms are continuous and correct + easily derived from point clouds + exact form of many geometric shapes + less storage; - intersections expensive to compute - cannot guarantee exact intersect - difficult to define a UV space.

\subsection{Learning 3D Implicit Shapes}

\textbf{Inference}: to get a mesh, sample points, predict occupancy/SDF, use marching cubes.

\subsubsection{From watertight meshes}
Sample points in space, compute GT occupancy/SDF, CE loss.

\subsubsection{From point clouds} Only have samples on the surface.
Weak supervision: loss = $|f_\theta(x_i)|^2 + \lambda \E_x(\norm{\nabla_x f_\theta(x)} - 1)^2$,
edge points should have $\norm{\nabla f} \approx 1$ by def. of SDF (Eikonal Equation), $f \approx 0$.

\subsubsection{From images} Need differentiable rendering 3D $\to$ 2D.
\textbf{Differentiable Volumetric Rendering}: for a point conditioned on encoded image,
predict occupancy $f(x)$ and RGB color $c(x)$.
\textbf{Forward}: for a pixel, raymarch and root find $\hat{p} : f(\hat{p}) = 0$ with secant. Set pixel color to $c(\hat{p})$.
\textbf{Backward}: see proofs.

\subsection{Neural Radiance Fields (NeRF)} \phantom{a}

$(x, y, z, \theta, \phi) \xrightarrow{\textsc{nn}} (r, g, b, \sigma)$.
Density $\sigma$ is predicted before adding view direction $\theta, \phi$ to limit effects of viewing angle on geometry,
then one layer for color.
\textbf{Forward}: shoot ray, sample points along it and blend:
$\alpha_i \coloneqq 1 - \exp( - \sigma_i \delta_i), \delta_i \coloneqq t_{i+1} - t_i,
T_i \coloneqq \prod_{j = 1}^{i-1}(1 - \alpha_j)$,
color is $c = \sum_i T_i \alpha_i c_i$.
Optimized on many views of the scene.
Can handle transparency/thin structure,
but worse geometry.
Needs many (50+) views for training, slow rendering for high res,
only models static scenes.

\subsubsection{Positional Encoding for High Frequency Details}
Replace $x, y, z$ with pos. enc. or rand. Fourier feats.
Adds high frequency feats.

\subsubsection{NeRF from sparse views} Regularize geometry and color.

\subsubsection{Fast NeRF render. and train.}
Replace deep MLPs with learn. feature hash table + small MLP.
For $x$ interp. features between corners.

\subsubsection{SNARF}
Model implicit function in canonical space.

\subsubsection{Vid2Avatar} Human in a sphere, background with NeRF, model body in canonical space.

\subsection{3D Gaussian Splatting}

\textbf{Alternative parametr.}:
Find a cover of object with primitives, predict inside.
Or sphere clouds. Both ineff. for thin structures.
Ellipsoids are better.

Initialize point cloud randomly or with an approx. reconstruction.
Each point has a 3D Gaussian.
Use camera params. to project (``splat'') Gaussians to 2D
and differentiably render them.
Adaptive density control moves/clones/merges points.

Rasterization: for each pixel sort Gaussians by depth, opacity
$\alpha = o \cdot \exp( - 0.5(x - \mu')\tran \Sigma'{}^{-1}(x - \mu'))$,
rest same as NeRF.

Each Gaussian primitive has a center $\mu \in \R^3$, a covariance $\Sigma \in \R^{3\times 3}$, and a color $c \in \R^3$ and an opaciti $o \in \R$.

To model view-dependent color, the color can be replaced with spherical harmonics, i.e. $c \in \R^9$

To keep covariance semi-definite: $\Sigma = RSS^\top R^\top$, where $S \in \R^{3\times 3}$ is a diagonal scaling matrix and $R \in \R^{3\times 3}$ is a rotation matrix.

%\includesvg[width=\linewidth]{figures/splatting.svg}





\section{Proofs}

\subsection*{Softmax derivative}

Let $\hat{y}_i = f(x)i = \frac{\exp(x_i)}{\sum_{j = 1}^d \exp(x_j)}, x \in \R^d$,
$y$ is 1-hot $\in \R^d$,
negative log-likelihood
$L(\hat{y}, y) = - \sum_{i = 1}^d y_i \log \hat{y}_i$.

$\pdv{L}{x_i} = \pdv{L}{\hat{y}} \pdv{\hat{y}}{x_i}.\sep$
$\pdv{\hat{y}_i}{x_i} = \frac{\exp(x_i) \sum_{j = 1}^d \exp(x_j) - \exp^2(x_i)}{\left(\sum_{j = 1}^d \exp(x_j)\right)^2}
= \frac{\exp(x_i)}{\sum_{j = 1}^d \exp(x_j)} \left(\frac{\sum_{j = 1}^d \exp(x_j)}{\sum_{j = 1}^d \exp(x_j)} - \frac{\exp(x_i)}{\sum_{j = 1}^d \exp(x_j)}\right) = \hat{y}_i(1 - \hat{y}_i). \sep$
$\pdv{\hat{y}_k}{x_i} = \frac{ - \exp(x_i)\exp(x_k)}{\left(\sum_{j = 1}^d \exp(x_j)\right)^2} = - \hat{y}_i \hat{y}_k.\sep$
$\pdv{L}{\hat{y}_k} = - \frac{y_k}{\hat{y}_k}. \sep$
$\pdv{L}{x_i} = - \frac{y_i}{\hat{y}_i} (\hat{y}_i(1 - \hat{y}_i)) - \sum\limits_{k \neq i} \frac{y_k}{\hat{y}_k}( - \hat{y}_i \hat{y}_k) = - y_i + y_i \hat{y}_i + \sum\limits_{k \neq i} y_k \hat{y}_i
= - y_i + \hat{y}_i \sum_k y_k = \hat{y}_i - y_i$.

\subsection*{BPTT}

$\rho$ is the identity function, $\partial^+$ is the immediate derivative, ignoring the effect from recurrence.

$\pdv{\vh_t}{W} = \pdv{W} \vh_t(\rho(W), \vh_{t-1}(W))
= \pdv{\vh_t}{\rho} \pdv{\rho}{W} + \pdv{\vh_t}{\vh_{t-1}} \pdv{\vh_{t-1}}{W}
= \pdv{{}^+ \vh_t}{W} + \pdv{\vh_t}{\vh_{t-1}} \pdv{\vh_{t-1}}{W}
= \pdv{{}^+ \vh_t}{W} + \pdv{\vh_t}{\vh_{t-1}} \left[\pdv{{}^+ \vh_{t-1}}{W} + \pdv{\vh_{t-1}}{\vh_{t-2}}\pdv{{}^+ \vh_{t - 2}}{W} + \dots\right]
= \sum_{k = 1}^t \pdv{\vh_t}{\vh_k} \pdv{{}^+ \vh_k}{W} \sep \pdv{L_t}{W} = \pdv{L_t}{\hat{y}_t} \pdv{\hat{y}_t}{\vh_t} \sum_{k = 1}^t \pdv{\vh_t}{\vh_k} \pdv{{}^+ \vh_k}{W}$

\subsection*{BPTT divergence}

Let $\lambda_1$ be the largest singular value of $\mW_{hh}$,
$\norm{\diag f'(\vh_{i-1})} < \gamma, \gamma \in \R, \norm{\cdot}$ is the spectral norm.
If $\lambda_1 < \gamma^{-1}$, then
$\forall i \ \ \norm{\pdv{\vh_i}{\vh_{i-1}}} \leq \norm{\mW_{hh}\tran} \norm{\diag f'(\vh_{i-1})} < \frac{1}{\gamma}\gamma < 1$
$\Rightarrow \exists \eta : \forall i \ \ \norm{\pdv{\vh_i}{\vh_{i-1}}} \leq \eta < 1$,
by induction over $i$: $\norm{\prod_{i = k + 1}^t \pdv{\vh_i}{\vh_{i-1}}} \leq \eta^{t - k}$,
so the gradients vanish as $t \to \infty$.
Similarly, if $\lambda_1 > \gamma^{-1}$, then gradients explode.

\subsection*{$\kl{\cdot}{\cdot} \geq 0$}

$-\kl{p}{q} = -\E_{x \sim p} \log \frac{p(x)}{q(x)} = \E \log \frac{q(x)}{p(x)} \leq \log \E_{x \sim p} \frac{q(x)}{p(x)}
= \log \int q(x) \dd x = \log 1 = 0$.

\subsection*{VAE ELBO}

\def\ith{^{(i)}}
$\log p_\theta(x\ith)
= \E_{z \sim q_\phi(z \mid x\ith)} \log p_\theta(x\ith)
= \E_z \log \frac{p_\theta(x\ith \mid z) p_\theta(z)}{p_\theta(z \mid x\ith)}
= \E_z \log \frac{p_\theta(x\ith \mid z) p_\theta(z) q_\phi(z \mid x\ith)}{p_\theta(z \mid x\ith) q_\phi(z \mid x\ith)}
= \E_z \log p_\theta(x\ith \mid z) - \E_z \log \frac{q_\phi(z \mid x\ith)}{p_\theta(z)} + \E_z \log \frac{q_\phi(z \mid x\ith)}{p_\theta(z \mid x\ith)}
= \E_z \log p_\theta(x\ith \mid z) - \kl{q_\phi(z \mid x\ith)}{p_\theta(z)} + \kl{q_\phi(z \mid x\ith)}{p_\theta(z \mid x\ith)}$.

\subsection*{KL for ELBO} Let $p(z) = \mathcal{N}(0, \mI), q(z\mid x) = \mathcal{N}(\mu, \sigma^2 \mI)$, $J \coloneqq \dim z$.
By $\int p(z) \log q(z) \dd z = - \frac{J}{2} \log 2\pi - \frac{1}{2} \sum_{j = 1}^J \log\sigma^2_{q, j}
- \frac{1}{2} \sum_{j = 1}^J \frac{\sigma_{p, j}^2 + (\mu_{p, j} - \mu_{q, j})^2}{\sigma^2_{q, j}}$
we have $\int q(z\mid x) \log p(z) \dd z = - \frac{J}{2} \log 2\pi - \frac{1}{2} \sum_{j = 1}^J(\sigma_j^2 + \mu_j^2)$
and $\int q(z\mid x) \log q(z\mid x) \dd z = \frac{J}{2} \log 2\pi - \frac{1}{2} \sum_{j = 1}^J(\log \sigma_j^2 + 1)$,
so $ - \kl{q(z\mid x)}{p(z)} = \frac{1}{2} \sum_{j = 1}^J (1 + \log \sigma_j^2 - \mu_j^2 - \sigma_j^2)$

\subsection*{Optimal discriminator} $D^*$ maximizes $V(G, D) = \int_x p_d \log D(x) \dd x + \int_z p(z) \log(1 - D(G(Z))) \dd z
= \int_x p_d \log D(x) \dd x + p_m(x) \log(1 - D(x)) \dd z$,
and for $f(y) = a\log(y) + b\log(1 - y):$
$f'(y) = \frac{a}{y} - \frac{b}{1 - y} \Rightarrow f'(y) = 0 \Leftrightarrow y = \frac{a}{a + b}$,
$f''(\frac{a}{a + b}) = - \frac{a}{\left(\frac{a}{a + b}\right)^2} - \frac{b}{\left(1 - \frac{a}{a + b}\right)^2} < 0$
for $a, b > 0 \Rightarrow $ max. at $\frac{a}{a + b}$ $\Rightarrow$ $D^* = \frac{p_d(x)}{p_d(x) + p_m(x)}$

\subsection*{Expectation of reparam.}

\!\!$\nabla_\varphi \E_{p_\varphi(z)}(f(z)) = \nabla_\varphi \int p_\varphi(z) f(z) \dd z
= \nabla_\varphi\! \int\!\! p_\varphi(z) f(z) \dd z \!=\! \nabla_\varphi\! \int\!\! p_\varphi(z) f(g(\epsilon, \varphi)) \dd \epsilon
\!=\! \E_{p(\epsilon)}\! \nabla_\varphi f(g(\epsilon, \varphi))$

\subsection*{$q(x_t \mid x_0)$}
$\vx_t = \sqrt{1 - \beta_t} \vx_{t-1} + \sqrt{\beta_t} \epsilon
= \sqrt{\alpha_t} \vx_{t-1} + \sqrt{1 - \alpha_t} \epsilon
= \sqrt{\alpha_t \alpha_{t - 1}} \vx_{t-2} + \sqrt{1 - \alpha_t \alpha_{t - 1}} \epsilon
= \dots = \sqrt{\overline{\alpha}_t} \vx_0 + \sqrt{1 - \overline{\alpha}_t} \epsilon$


\subsection*{Implicit differentiation}

$\dv{y}{x}$ of $x^2 + y^2 = 1$:

$\dv{x}(x^2 + y^2) = \dv{x}(1) \Rightarrow \dv{x}x^2 + \dv{x}y^2 = 0 \Rightarrow 2x + (\dv{y} y^2) \dv{y}{x} = 0$
$\Rightarrow 2x + 2y \dv{y}{x} = 0 \Rightarrow \dv{y}{x} =- \frac{x}{y}$

\subsection*{DVR Backward pass}
$\pdv{L}{\theta} = \sum_u \pdv{L}{\hat\mI_u} \cdot \pdv{\hat\mI_u}{\theta} \sep\!$
$\pdv{\hat\mI_u}{\theta} = \pdv{c_\theta(\hat\vp)}{\theta} + \pdv{t_\theta(\hat\vp)}{\hat\vp}
 \cdot \pdv{\hat\vp}{\theta}$.
Ray $\hat\vp = r_0 + \hat{d}\vw$,
$r_0$ is camera pos., $\vw$ is ray dir., $\hat{d}$ is ray dist.
Implicit def.: $f_\theta(\hat\vp) = \tau$.
Diff.:
$\pdv{f_\theta(\hat\vp)}{\theta} + \pdv{f_\theta(\hat\vp)}{\hat\vp} \cdot \pdv{\hat\vp}{\theta} = 0 \Rightarrow
\pdv{f_\theta(\hat\vp)}{\theta} + \pdv{f_\theta(\hat\vp)}{\hat\vp} \cdot \vw\pdv{\hat{d}}{\theta} = 0 \Rightarrow 
\pdv{\hat\vp}{\theta} = \vw \pdv{\hat{d}}{\theta} = - \vw(\pdv{f_\theta(\hat{\vp})}{\hat{\vp}} \cdot \vw)^{-1}
\pdv{f_\theta(\hat\vp)}{\theta}$

\section{Appendix}

\subsection*{Secant Method}

Line $(x_0, f(x_0)) \to (x_1, f(x_1))$, approx.:
$y = \frac{f(x_1) - f(x_0)}{x_1 - x_0}(x - x_1) + f(x_1)$,
$y = 0$ at $x_2 = x_1 - f(x_1) \frac{x_1 - x_0}{f(x_1) - f(x_0)}$.
Approximates Newton's method without derivatives.

\subsection*{Implicit plane from 3 points}

$(x_1, 0, 0), (0, y_1, 0), (0, 0, z_1) \Rightarrow x / x_1 + y / y_1 + z / z_1 - 1 = 0$.
More generally: let $a, b$ any vectors on plane,
$n \coloneqq a \times b = (a_2 b_3 - a_3 b_2, a_3b_1 - a_1b_3, a_1b_2 - a_2b_1) \Rightarrow n_1 x + n_2 y + n_3 z + k = 0$,
subst. any point to find $k$.

\subsection*{Torus equation}

\!\!\!\!$(\sqrt{x^2 + y^2} - R)^2 + z^2 = r^2$, cent. $0$, around $z$ axis.

\subsection*{Chain rule}
$\pdv{z_k}{x_j} = \sum_{i=1}^{m} \pdv{z_k}{y_i} \pdv{y_i}{x_j} \sep$
$\pdv{z}{x} = \pdv{z}{y} \pdv{y}{x}$


\subsection*{Derivatives}

$(f \cdot g)' = f'g + fg'$,
$(f / g)' = (f'g - fg') / g^2$,
$(f \circ g)' = f'(g)g'$,
$(f^{-1})' = 1 / f'(f^{-1})$,
$(\log x)' = 1 / x$.

\subsection*{Linear algebra}

Matrix det lemma: $\det(\mA + \vu\vv\tran) = (1 + \vv\tran \mA^{-1}\vu) \det \mA$

\subsection*{Jensen's inequality}
$f(\E[X]) \leq \E[f(X)]$ if $f$ is convex, i.e. $\forall t \in [0, 1], x_1, x_2 \in X : f(tx_1 + (1 - t)x_2) \leq tf(x_1) + (1 - t)f(x_2)$.
The opposite is true for concave functions (e.g. $\log$).


\subsection*{Gaussians}

$\mathcal{N}(\mu_1, \Sigma_1) + \mathcal{N}(\mu_2, \Sigma_2)
= \mathcal{N}(\mu_1 + \mu_2, \Sigma_1 + \Sigma_2)$,

$a \cdot \mathcal{N}(\mu, \Sigma) = \mathcal{N}(a\mu, a^2\Sigma)$.

$\mathcal{N} = \frac{1}{\sqrt{(2\pi)^d |\Sigma|}} \exp( - \frac{1}{2} (x - \mu)\tran \Sigma^{-1} (x - \mu))$

\subsection*{VRNN}

$p_\theta(\vz) = \prod_{t = 1}^T p_\theta(z_t \mid \vz_{ < t}, \vx_{ < t})$,

$q_\phi(\vx \mid \vz) = \prod_{t = 1}^T q_\phi(z_t \mid \vx_{ \leq t}, \vz_{ < t})$,

$p_\theta(\vx, \vz) = \prod_{t = 1}^T p_\theta(x_t \mid \vz_{ \leq t}, \vx_{ < t})p_\theta(z_t \mid \vz_{ < t}, \vx_{ < t})$.

\subsection*{Misc}
A \textbf{translation vector} is added.

\textbf{Bayes rule}:
$P(A \mid B) = P(B \mid A)P(A) / P(B)$.

A function $f$ is \textbf{volume preserving} if $\left|\det \pdv{f^{-1}(x)}{x}\right| = 1$.

\textbf{Negative log-likelihood} $L(\hat{y}, y) = - \sum_i y_i \log \hat{y}_i$

\end{multicols*}

\end{document}